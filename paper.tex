\documentclass[12pt,twoside,a4paper]{article}
\renewcommand*\familydefault{\sfdefault}
\usepackage[utf8]{inputenc}
\usepackage[T1]{fontenc}
\usepackage[ngerman]{babel}
\usepackage[left=3cm,right=3cm,top=2cm,bottom=4cm]{geometry}
\usepackage{graphicx}
\usepackage{helvet}
\renewcommand{\familydefault}{\sfdefault}
\linespread{1.25}
\usepackage[table]{xcolor}
\usepackage{hyperref}
\hypersetup{
    colorlinks,
    linkcolor={black!},
    citecolor={blue!50!black},
    urlcolor={blue!50!black}
}
\definecolor{light-yellow}{RGB}{255, 255, 204}
\usepackage[numbib,nottoc]{tocbibind}
\usepackage{caption}
\usepackage{verbatimbox}
\usepackage{graphicx}
\usepackage{apacite}
\usepackage[acronym]{glossaries}
\makeglossaries
\newacronym{erm}{ERM}{Entity Relationship Model}
\newacronym{uml}{UML}{Unified Markup Language}
\begin{document}
\pagenumbering{gobble}
% Deckblatt
\begin{center}
\href{https://www.intension.de/}{\includegraphics[width=6cm]{images/intension}}\hfill\href{https://www.dhbw-stuttgart.de}{\includegraphics[width=4cm]{images/dhbw}}\\
\large
\vspace{3cm}
\textbf{Featherkraken}: Bestpreissuche für Flugangebote mit variablen Abflughäfen\\
\vspace{2cm}
\includegraphics[width=4cm]{images/featherkraken}\\
\textbf{\Large STUDIENARBEIT}
\vspace{1cm}
\\des Studienganges Informatik
\\an der Dualen Hochschule Baden-Württemberg Stuttgart
\\von
\\Ingo Kuba
\end{center}
\vspace{2cm}
\textbf{Matrikelnummer, Kurs}\hfill place, holder\\
\textbf{Ausbildungsfirma}\hfill intension GmbH\\
\textbf{Betreuer}\hfill Place Holder
\newpage
% Pre-Einleitung
\section*{Erklärung zur Eigenleistung}
Hiermit erkläre ich, dass ich die vorliegende Studienarbeit selbständig verfasst und keine anderen als die angegebenen Hilfsmittel benutzt habe.\\
Die Stellen der Studienarbeit, die anderen Quellen im Wortlaut oder dem Sinn nach entnommen wurden, sind durch Angaben der Herkunft kenntlich gemacht. Dies gilt auch für Zeichnungen, Skizzen, bildliche Darstellungen sowie für Quellen aus dem Internet.
\vspace{1cm}\\Ostfildern, den \today \hspace{1cm} \hrulefill
\newpage
% Abstract
\section*{Zusammenfassung}
% Verzeichnisse
\newpage
\tableofcontents
\newpage
\listoffigures
\newpage
\printglossary[type=\acronymtype]
\printglossary
\newpage
\interlinepenalty=10000
\pagenumbering{arabic}
\setcounter{page}{1}
% Einleitung:
\section{Einleitung}
\subsection{Motivation}
In der Regel möchte ein Fluggast den günstigsten Preis für eine bestimmte Route A nach B. Jede Flugsuchmaschine im Internet bietet diese Feature. Manchmal sucht ein Fluggast auch einfach nach Inspiration und möchte Angebote von A nach X, wobei X variabel ist. Einige Suchmaschinen bieten diese Suche bereits an. Worum es in dieser Studienarbeit geht, ist der umgekehrte Fall: X nach B. Also von welchem beliebigen Flughafen kommt man möglich günstig an ein festes Ziel.\newline
Gerade auf hochpreisigen Strecken kann es sich lohnen einen Umweg zu fliegen.
\subsection{Aufgabenstellung}
Bei der Suche sollen die klassischen Filterkriterien implementiert werden. Das heißt die Unterscheidung ob man nur einen Hinflug oder Hin- und Rückflug buchen möchte. Des weiteren soll man jeweils ein Datum für An- und Abreise festlegen können, welches um drei Tage flexibel sein soll. Neben der Buchungsklasse (Economy, Business, First Class) soll auch die Wahl der Airline oder Allianz eingeschränkt werden können. Außerdem soll man Passagier- und Umsteigeanzahl wählen können.\newline
Zusätzlich soll ein Entfernungsfilter um einen möglichen Abflughafen bereitgestellt werden. Zum Beispiel wird nur nach Angeboten gesucht, bei dem sich der Startflughafen maximal 800km (Entfernungsfilter) vom Flughafen Stuttgart (möglicher Abflughafen) entfernt befindet.\newline
Diese Flugsuchmaschine soll über ein Web-Frontend vom Nutzer bedient werden können. \cite{example}
\section{Grundlagen}
\subsection{Entity Relationship Model}
Um das Datenmodell der Anwendung darzustellen wurde eine vereinfachte Variante des \acrfull{erm} verwendet.\\
\subsection*{Entities}
Ein Objekt oder Entity wird in einem Rechteck dargestellt und kann Attribute besitzen, wobei komplexe Attribute als Beziehungen zu anderen Objekten dargestellt werden. Der Name der Beziehung entspricht hierbei dem Attributnamen im Code. Die Anzahl der möglichen Relationen wird in \acrshort{uml}-Notation angegeben. Zum Beispiel ist in Abbildung \ref{fig:erm-entity} zu sehen, dass eine Person null bis n Autos besitzen kann.
\begin{center}
	\captionsetup{type=figure}
	\includegraphics[width=\textwidth]{images/ERM-Entity}
	\captionof{figure}[Beispiel für eine Entity im \acrshort{erm}]{Beispiel für eine Entity mit Attributen und einer Beziehung}
	\label{fig:erm-entity}
\end{center}
\subsection*{Attribute}
Attribute können dabei eindeutig, optional oder mehrwertig sein. Die Unterscheidung zwischen Datum, Zahl oder Zeichenkette wird in einem \acrlong{erm} nicht dargestellt.
\begin{center}
	\captionsetup{type=figure}
	\includegraphics[width=10cm]{images/ERM-Attributes}
	\captionof{figure}[Verschiedene Arten von Attributen im \acrshort{erm}]{Attribute (v.l.): eindeutig, optional und mehrwertig}
	\label{fig:erm-attributes}
\end{center}
% Hauptteil
\section{Entwurf}
\subsection{Auswahl der externen API}
Um Flugdaten zu erhalten muss eine externe Schnittstelle benutzt werden, welche die Pflichtanforderungen erfüllt und dabei leicht zu verwenden ist. Die Wahl der Schnittstelle fiel dabei auf eine Rest-API, da diese sehr einfach zu benutzen sind. Für die Auswahl des Anbieters wurde eine Tabelle\textsuperscript{\ref{fig:api-comparison}} erstellt, welche die Pflichtanforderungen mit den Funktionalitäten der jeweiligen Schnittstelle abgleicht. Dabei erfüllte die API von Kiwi nicht nur alle Anforderungen, sondern war auch sehr gut dokumentiert und mit Beispielen beschrieben. Des Weiteren bot Kiwi auch eine Rest-API um Flughäfen im Umkreis gegebener Koordinaten zu finden, was der Aufgabe entgegen kam.
\begin{center}
	\captionsetup{type=figure}
	\resizebox{\textwidth}{!}
	{\begin{tabular}{ l | c | c | c | c | c | c }
		& \textbf{Skyscanner} & \textbf{Hipmunk} & \textbf{Kajak} & \textbf{Flight Data} & \textbf{Flight Bookings} & \textbf{Kiwi Flights}\\
		\hline
		\textbf{Single flight} & \cellcolor{green!50}yes & \cellcolor{green!50}yes & \cellcolor{green!50}yes & \cellcolor{green!50}yes & \cellcolor{green!50}yes & \cellcolor{green!50}yes\\
		\hline
		\textbf{Two directions} & \cellcolor{red!75}no & \cellcolor{red!75}no & \cellcolor{red!75}no & \cellcolor{green!50}yes & \cellcolor{red!75}no & \cellcolor{green!50}yes\\
		\hline
		& & & & & &\\
		\hline
		\textbf{Specific date} & \cellcolor{green!50}yes & \cellcolor{green!50}yes & \cellcolor{green!50}yes & \cellcolor{green!50}yes & \cellcolor{green!50}yes & \cellcolor{green!50}yes\\
		\hline
		\textbf{Flexible date} & \cellcolor{red!75}no & \cellcolor{yellow!75}limited & \cellcolor{yellow!75}limited & \cellcolor{green!50}yes & \cellcolor{red!75}no & \cellcolor{green!50}yes\\
		\hline
		& & & & & &\\
		\hline
		\textbf{Class} & \cellcolor{green!50}yes & \cellcolor{green!50}yes & \cellcolor{green!50}yes & \cellcolor{yellow!75}limited & \cellcolor{green!50}yes & \cellcolor{green!50}yes\\
		\hline
		\textbf{Passengers} & \cellcolor{green!50}yes & \cellcolor{green!50}yes & \cellcolor{green!50}yes & \cellcolor{red!75}no & \cellcolor{green!50}yes & \cellcolor{green!50}yes\\
		\hline
		\textbf{Airline} & \cellcolor{green!50}yes & \cellcolor{red!75}no & \cellcolor{red!75}no & \cellcolor{red!75}no & \cellcolor{red!75}no & \cellcolor{green!50}yes\\
		\hline
		\textbf{Stops} & \cellcolor{red!75}no & \cellcolor{red!75}no & \cellcolor{red!75}no & \cellcolor{red!75}no & \cellcolor{red!75}no & \cellcolor{green!50}yes
	\end{tabular}}
	\captionof{figure}[Vergleich der APIs]{Tabelle zum Vergleich der APIs}
	\label{fig:api-comparison}
\end{center}
\newpage
\subsection{Datenmodell}
Das \acrlong{erm} für das Datenmodell wurde hier aufgeteilt in Suchanfrage (\texttt{SearchRequest}) und Suchergebnis (\texttt{SearchResult}).
\subsubsection{SearchRequest}
Eingehende Anfragen an den Service (\texttt{SearchRequest}) haben verschiedene Parameter, welche die zuvor genannten Pflichtanforderungen abdecken. Komplexe Attribute wie die Zeitspanne (\texttt{Timespan}) für An- und Abreise sowie den Ursprungs- und Ziel-Flughafen (\texttt{Airport}) wurden in eigene Objekte ausgelagert, damit sie so wiederverwendet werden können.
\begin{center}
	\captionsetup{type=figure}
	\includegraphics[width=\textwidth]{images/datamodel-SearchRequest}
	\captionof{figure}[\acrshort{erm} SearchRequest]{\acrlong{erm} des SearchRequest Objekts}
\end{center}
\subsubsection{SearchResult}
Antworten des Service (\texttt{SearchResult}) sind etwas komplexer als Anfragen aufgebaut, teilen jedoch manche Attribute beziehungsweise Objekte mit diesen. So enthält ein Suchergebnis eine Ansammlung von sogenannten \texttt{Trips}, welche sich aus Flügen (\texttt{Flight}) der An- und Abreise zusammensetzen. Diese Flügen haben widerrum selbstverständlich selbst jeweils einen Start- und Zielflughafen, welche in dem \texttt{Route}-Objekt mit Informationen zu Abflug- und Ankunftzeiten sowie der Airline enthalten sind.\\
Bei erster Betrachtung fällt auf, dass Informationen über Airlines und Zeiten redundant sind, was jedoch für die optimale Anzeige in der Oberfläche gedacht ist.
\begin{center}
	\captionsetup{type=figure}
	\includegraphics[width=\textwidth]{images/datamodel-SearchResult}
	\captionof{figure}[\acrshort{erm} SearchResult]{\acrlong{erm} des SearchResult Objekts}
\end{center}
\newpage
\subsection{Framework}
Die Anwendung wird in eine Server- und eine Web-Anwendung unterteilt, für welche jeweils eine Framework-Technologie gewählt werden muss. Die Serveranwendung übersetzt Suchanfragen von der Oberfläche zu einem Format, welches die externe Schnittstelle akzeptiert und leitet das Sucherergebnis daraufhin an die Oberfläche zurück. Dies bedeutet, dass die Serveranwendung als Rest-Client für die externe Schnittstelle und selbst als Rest-Schnittstelle für die Oberfläche dient.\\
Für die Komponenten war das Thema Plattformunabhängigkeit sehr wichtig und aus eigener Erfahrung eignet sich dafür Java sehr gut, da es eine große Community hat und eine Anwendung in kurzer Zeit aufgesetzt ist. Außerdem bietet Java einfache Möglichkeiten eine externe Rest-Schnittstelle anzusprechen. Als Bedingung für die Web-Oberfläche ist es unabdingbar, dass die Anwendung intuitiv bedienbar ist und auf allen Endgeräten gut dargestellt werden kann. Für diesen Zweck kommen eigentlich nur drei Frameworks in Frage: Angular, React.js und Vue.js. Dabei wurde React gewählt, welches sich besser als Vue.js und Angular kleine Anwendungen eignet, wie es in diesem Fall zutrifft. Bei dem hier vorliegenden Fall wird es sogar eine sogenannte Single-Page-Webapplication, das heißt die gesamte Funktion kann auf einer Seite dargestellt werden.
% Umsetzung
\section{Implementierung}
Wie portabel ist das System? -> Docker, React -> static html+js
\section{Testing}
Unit tests?
How to test external api?
Aspects of an API: speed? maybe Beständigkeit in der Speed?
% Ende
\section{Zusammenfassung}
\subsection{Ausblick}
-> mehr Schnittstellen anbinden (ist schon bereit dafür)
-> jedoch nicht async
-> Monitor speed of my requests!!
% Literaturverzeichnis
\newpage
% set interlinepenalty to not split entries on page break
\bibliographystyle{apacite}
\bibliography{paper}
\end{document}